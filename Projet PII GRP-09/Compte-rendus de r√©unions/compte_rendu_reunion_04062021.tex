% \documentclass[12pt]{article}
% \usepackage[french]{babel}
% \usepackage{natbib}
% \usepackage[utf8]{inputenc}
% \usepackage[T1]{fontenc}
% \usepackage{tikz}
% \usepackage{amsmath}
% \usepackage{graphics}
% \usepackage{graphicx}
% \usepackage{url}
% \usepackage{psfrag}
% \usepackage{fancyhdr}
% \usepackage{vmargin}
% \usepackage[backend=biber]{biblatex}
% \usepackage{csquotes}
% \usepackage[hidelinks]{hyperref}
% \usepackage{enumitem}

% \pagestyle{fancy}


% \begin{document}
\subsubsection*{\large{Réunion d'équipe du 4 juin 2021}}
    \addcontentsline{toc}{subsubsection}{Réunion d'équipe du 4 juin 2021}
\begin{center}
\begin{tabular}{| l | l || c | c |}
    \hline
    Membres présents & Membres absents & Durée & Lieu \\
    \hline
    Maël SAILLOT & & & \\ Céline ZHANG & & 1h30 & Discord \\ Ahmed ZIANI & & & \\
    \hline
\end{tabular}
\end{center}

\subsubsection*{Ordre du jour}
\begin{enumerate}
    \item Avancement des tâches
    \item Discussion générale et amélioration du travail actuel
    \item Prochaines tâches
\end{enumerate}

\subsubsection*{Avancement des tâches}
\paragraph{Maël SAILLOT} a regroupé les scripts de remplissage pour en faire un seul script optimisé, a commencé le script de test de cohérence entre les fichiers et a mis à jour le schéma de la base de données.
\paragraph{Céline ZHANG} a fait le template du rapport, a fini l'écriture de l'introduction et de la gestion de projet, a ajouté les comptes rendus et les mentions légales, ainsi que d'autres annexes.
\paragraph{Ahmed ZIANI} a fait la matrice \textsf{SWOT}, \textsf{RACI} et a fait les premières fonctionnalités de l'application de web tels que l'affichage des données d'un candidat, la liste des épreuves, des options, des professions, des classements, des notes, etc.

\subsubsection*{Discussion générale et amélioration du travail actuel}
Il faudra vérifier les fichiers, et penser à faire un script de test de remplissage. Il manque pour l'application les fonctionnalités de mises en stats, d'affichage de graphes ou de cartes et les interactions permettant de rendre l'application plus facile d'utilisation. 

\subsubsection*{Prochaines tâches}
L'équipe devra écrire l'état de l'art, finir le script de test de cohérence, ajouter les fonctionnalités stats, cartographique, les styles et l'interface.


\paragraph{\emph{TO-DO LIST}}
\begin{itemize}
    % \item Finir l'application web
    \item Écrire l'état de l'art
    \item Finir le script de test de cohérence
    \item Ajouter les fonctionnalités stats
    \item Ajouter du style et une interface
    \item Ajouter la cartographie et éventuellement des graphes
    % \item Écrire l'explication des algorithmes
    % \item Ajouter les commentaires sur le code
    % \item Mettre à jour le \textsf{README.md}
    % \item Réfléchir à la présentation
    
\end{itemize}

%\emph{Prochaine réunion : 09/06/2021}\\

% \end{document}