% \documentclass[12pt]{article}
% \usepackage{natbib}
% \usepackage[french]{babel}
% \usepackage[utf8]{inputenc}
% \usepackage[T1]{fontenc}
% \usepackage{tikz}
% \usepackage{amsmath}
% \usepackage{graphics}
% \usepackage{graphicx}
% \usepackage{url}
% \usepackage{psfrag}
% \usepackage{fancyhdr}
% \usepackage{vmargin}
% \usepackage[backend=biber]{biblatex}
% \usepackage{csquotes}
% \usepackage[hidelinks]{hyperref}
% \usepackage{enumitem}

% \pagestyle{fancy}


% \begin{document}
\subsubsection*{\large{Réunion d'équipe du 3 avril 2021}}
    \addcontentsline{toc}{subsubsection}{Réunion d'équipe du 3 avril 2021}
\begin{center}
\begin{tabular}{| l | l || c | c |}
    \hline
    Membres présents & Membres absents & Durée & Lieu \\
    \hline
    Maël SAILLOT & & & \\ Céline ZHANG & & 1h45 & Discord \\ Ahmed ZIANI & & & \\
    \hline
\end{tabular}
\end{center}

\subsubsection*{Ordre du jour}
\begin{enumerate}
    \item Mise à niveau générale sur les connaissances du sujet
    \item Organisation du projet
    \item Sélection des outils nécessaires au projet
    \item Mise en place de la gestion de projet
    \item Planification des prochaines réunions
\end{enumerate}

\subsubsection*{Mise à niveau générale sur les connaissances du sujet}
Nous avons mis en commun nos connaissances concernant le sujet, partagé nos documentations, et expliqué les notions ambigües. Ceci s'est suivi d'une présentation de nos préférences, de nos points forts et points faibles respectives.

\subsubsection*{Organisation du projet}
Nous avons parlé du sens de déroulement général du projet (quand faire les documentations, dans les premières lignes, quand aborder les parties, quand commencer la rédaction du rapport, etc.). Nous avons discuté d'une gestion de projet (les stratégies, les rôles, voir le tableau~\ref{tab:roles}, les facilités, les difficultés par la matrice de SWOT, voir figure~\ref{tab:swot}). Nous adoptons la gestion SCRUM, qui utilise la méthode du \textsl{sprint}\footnotemark \ pour réaliser les travaux.

    \begin{table}[!h]
    \begin{center}
        \begin{tabular}{|l|l|}
        \hline
            Membre de l'équipe 9 & Rôles ou charges \\
        \hline
        \hline
            Céline ZHANG & leader \\
        \hline
            Ahmed ZIANI & reviewer \\
        \hline
            Maël SAILLOT & responsable du git \\
        \hline
        \end{tabular}\\
        %\includegraphics[scale = 0.45]{Images/Gestion de Projet/Matrice_swot.png}
    \end{center}
    %\caption{Tableau des rôles}
    %\label{tab:roles}
    \end{table}

\paragraph{Céline ZHANG} se chargera de vérifier les tâches, les objectifs, de planifier les réunions et d'écrire les compte-rendus de réunions d'équipe.
\paragraph{Ahmed ZIANI} se chargera de revoir nos lignes de code (code review), de vérifier la cohérence et les tests.
\paragraph{Maël SAILLOT} s'occupera de gérer le dépôt git de l'équipe, de gérer les erreurs dû à des conflits.

\footnotetext{Méthode d'organisation de travail d'équipe qui consiste à se fixer des tâches pour un cycle (1 à 2 semaines) et de les finir avant la fin de celle-ci.}

\subsubsection*{Sélection des outils nécessaires au projet}
La présentation des outils utilisables et les outils nécessaires est primordiale pour commencer à travailler. Nous avons donc présenté les outils à disposition (\textsf{gitlab}, \textsf{Discord}, \textsf{SQLite}, \textsf{Overleaf}, \textsf{dbdiagram.io}, \textsl{C}, \textsf{Java}, \textsf{Python}, \textsf{html}).

\subsubsection*{Mise en place de la gestion de projet}
Nous avons choisi d'utiliser le tableau de bord Trello\footnotemark qui va nous permettre de mettre le cahier de charge en tâches nécessaires à la réalisation du livrable final. Ceci nous permet aussi de surveiller l'avancement de chacun sur ses tâches. Nous avons commencé par créer le Trello et y mettre des tâches.
\footnotetext{Outil qui nous permet de planifier en ligne des activités.\ref{fig:exTrello02062021}}

\subsubsection*{Planification des prochaines réunions}
Nous décidons de faire en général une réunion par semaine (flexible suivant les disponibilités), qui sera tous les dimanches à 17h00 sur Discord, bien sûr il peut y avoir des stand-up-meeting pour régler les soucis ou prendre des nouvelles (le tableau Trello permet aussi de laisser des messages pour indiquer les problèmes et demander de l'aide).

\paragraph{\emph{TO-DO LIST}}
\begin{itemize}
    \item Se documenter sur première partie du sujet concernant les notions de 3e forme normale, les contraintes d'intégrités
    \item Réfléchir à la structure de la base de données
    \item Faire un schéma liant les tables
\end{itemize}

\emph{Prochaine réunion : 06/04/2021}\\

% \end{document}